% Copyright © 2011-2013 Jesse DeGuire
%
% This file is part of Projector Controller.
%
% This work is licensed under the Creative Commons Attribution-ShareAlike 3.0 Unported License. You
% should have received a copy of the license along with this work.  If not, see
% <http://creativecommons.org/licenses/by-sa/3.0/>.
%
% File:   devguide.tex
% Author: Jesse DeGuire
%
% LaTeX source for the PJC Developers Guide document.  This can be made into a PDF using a suitable
% LaTeX distribution such as TeX Live for GNU/Linux or MiKTeX for Windows.

\documentclass{article}

\usepackage{anysize} 
\usepackage{graphicx}
\usepackage{wrapgfig} 
\usepackage[pdftex]{hyperref}
\hypersetup{colorlinks=true, urlcolor=cyan, linkcolor=blue}

\marginsize{1in}{1in}{1in}{1in}

% just so that the document revision is somewhere convenient
\newcommand{\docrev}{A}

% makes a backslash
\newcommand{\bs}{$\backslash$}


\begin{document}

% don't add extra space to sentence ends when justifying text
\frenchspacing

\noindent {\Huge Projector Controller} \\
{\large Developers Guide} Rev. \docrev \\
Author: Jesse DeGuire \\
Email: \href{mailto:jesse.a.deguire+pjc@gmail.com}{jesse.a.deguire+pjc@gmail.com} \\
Date: \today \\

\tableofcontents

% redefine \section command to start section on new page
\let\stdsection\section
\renewcommand{\section}{\newpage\stdsection}

% additional space between paragraphs
\addtolength{\parskip}{6pt}

\section*{Revision History}
\phantomsection
This section describes revisions made to this document.  Changes made to the hardware, firmware, or
software are described in their respective sections.

\begin {description}
  \item[Rev. A:] February 3, 2013\\
    Initial revision.
\end {description}

\section{Introduction}
This document provides implementation details for the various pieces of the Projector Controller
system.  The information contain within will hopefully be useful for anyone who wishes to modify or
tinker with the any part of the system or at least will satisfy an interested user's curiosity.
Anyone who is only looking to get the Controller running in his or her projector should instead
consult the \textit{Projector Controller Users Guide}.

This guide is divided up into a few sections.  The \hyperref[Hardware_sec]{Hardware} section
describes the components used on the board itself as well as the power requirements and capabilities
of the board.  The \hyperref[Firmware_sec]{Firmware} section contains information regarding the
program flow of the bootloader and application firmwares used by the Projector Controller.  This
also contains documentation for the command interfaces used by both to communicate to the host PC as
well as brief descriptions of many of the source code files.  Finally, the
\hyperref[Software_sec]{Software} section describes how the PC-side software communicates to the
board.  All sections will additionally state what tools were used in their development so that
anyone can build and modify the various pieces.

\subsection{Sources and Licenses} \label{SourceLic_sec}
All source files used to create the board, software, and firmwares are available in a Git repository
on GitHub at \url{https://github.com/jdeguire/pjcontroller}.  The repository also contains
datasheets for the hardware components and other reference material.

The firmware and software, the sources of which are located in the \texttt{firmware/} and
\texttt{software/} directories in the aforementioned repository, are made available under the GNU
Public License Version 3.  A copy of the GPLv3 is provided in the file \texttt{COPYING.txt} in the
repository root or online at \url{http://www.gnu.org/licenses/}.

This document, the \textit{Users Guide}, their associated \LaTeX{} source files, and the board
design files are all made available under the Creative Commons Attribution-ShareAlike 3.0 Unported
License.  The source files for the \textit{Developers Guide} and \textit{Users Guide} are available
in the \texttt{docs/tex/} directory.  The board design files are found in
\texttt{hardware/pjc-kicad/}.  A copy of the Creative Commons BY-SA 3.0 license is provided in the
file \texttt{CC BY-SA 3.0 legalcode.txt} in the repository root or online at
\url{http://creativecommons.org/licenses/by-sa/3.0/}.

Where attribution is required, it is sufficient to include a statement of reasonable visibility in
the derived work or supporting documentation mentioning that the work is derived from this project
and giving this author's name.  It would be appreciated if the statement also contained the URL,
given above, for the version control repository.

Located in \texttt{hardware/pjc-kicad/libs} are two files \texttt{SparkFun.lib} and
\texttt{SparkFun.mod}.  These were converted into their current form for use with KiCad from the
original Eagle libraries provided by SparkFun Electronics under the CC BY-SA 3.0 license.  The
latest such libraries can be found on GitHub at
\url{https://github.com/sparkfun/SparkFun-Eagle-Libraries}.

\subsection{Open Source Hardware}
\begin{wrapfigure}{r}{0.3\textwidth}
    \begin{center}
        \includgraphics[0.25\textwidth]{oshw-logo-200-px}
    \end{center}
\end{wrapfigure}

As described above, all design and source code files for Projector Controller are available under
licenses allowing one to ``study, modify, distribute, make, and sell the design or hardware based on
that design''.  The project was developed using freely-available tools, most of which are also open
source, lowering the barrier to accessing the design.  Therefore, the author believes that the
Projector Controller qualifies as Open Source Hardware as given in the Open Source Hardware
Definition 1.0.  The defintions is availabe from the Open Source Hardware Association at
\url{http://www.oshwa.org/definition/}.

Note that the logo itself does not have a license associated with it, but the guidelines presented
at \url{http://www.oshwa.org/faq/} recommded its use.  Any derived works which do not adhere to the
definition of Open Source Hardware should remove the logo from any files which contain it.


\section{Hardware} \label{Hardware_sec}
The hardware was designed using KiCad EDA, an open source (GPLv2) multi-purpose EDA tool available
for various flavors of GNU/Linux, Windows, and OS X.  Get KiCad from the official site at
\url{http://www.kicad-pcb.org/display/KICAD/Download+Kicad}.  If possible, download a Snapshot build
as those contain additional features over the rather old stable release.  Ubuntu users should use
the Adam Wolf PPA given on that page.

Datasheets for the various components used are available in the repository under
\texttt{hardware/datasheets}.  The datasheets are provided for reference only and belong to their
respective manufacturers.

\subsection{Design Considerations} \label{DesignConsid_ssec}
Becuase the Projector Controller will probably never acheive high-volume production, the main
consideration was ease of assembly.  When possible, the components used are of a large form factor
(such as 1206 for resistors) and the footprints for surface-mount devices use long pads to make
accessing them with a soldering iron easier.  The board uses a socket for the 28-pin DIP
microcontroller to allow easy replacement.  These choices allow the board to be assembled by hand
for small production runs and the allow user to more easily tinker with the board.

Other considerations were size of the board itself and price, which are related since larger PCBs
are more expensive.  These may seem to conflict with the previous consideration, but the Controller
is relatively simple, making use of not many components and reusing multiple of the same component
when possible (the board uses multiple 10k\Omega resistors, for example).  Fewer components also
means a smaller board size and less assembly.  A smaller board size means that the device will be
easier to install into one's projector.

Though not a primary consideration, the board was also designed with an eye on power consumption.
The board will be able to run certain componenets from USB power.  USB provides 5V at 100mA by
default, though a device can request up to 500mA.  The USB interface device does not identify itself
as a high-power device by default, so all 5V components will consume less than 100mA combined.  The
power consumption of the board is characterized in the section \hyperref[PowerUsage_ssec]{Power
 Usage}.

\subsection{Circuit Details} \label{CircuitDetails_ssec}
Following are desciptions of the different circuit pieces developed for the board.  The reader of
this section may want to have the schematic handy in order to follow along.  The titles of the
descriptions will closely match the text titles used in the schematic for the different circuits.

\subsubsection{Microcontroller} \label{Microcontroller_sssec}
The heart of the Projector Controller is an Atmel ATMega328p 8-bit microcontroller.  The MCU has
32KB of flash, 4KB of which is used for the bootloader, 2KB of RAM and several useful peripherals,
such as 10-bit ADCs, SPI, PWM outputs, timers, and quite a few IO pins.  The MCU is a 28-pin DIP
package which will insert into a socket on the board, making removal and replacement easy.

A Fox 20MHz crystal oscillator provides the CPU clock.

Three LEDs--red, yellow, and green--are provided to allow the firmware to convey diagnostic
information while running.  The LED outputs also control low-side drivers, which is explained in the
\hyperref[RelayLED_sssec]{Relay and LED Outputs} description.

A jumper, pulled low, is connected to an IO pin on the microcontroller.  Shorting the jumper sets
the state of that IO pin high.  The state of this pin is read by the bootloader when it first starts
up.  If the pin state is high, the bootloader will not start the application and will instead stay
in the bootloader.  This allows one to recover from a problematic application that fails on startup.

\subsubsection{USB/UART IO} \label{USB_UART_sssec}


\end{document}